\section{Xấp xỉ mô hình bậc cao theo phương pháp Skogestad (luật chia đôi)}

    Cho mô hình của đối tượng có dạng như sau:

    \begin{align*}
        G(s) = \dfrac{\displaystyle k \prod_{i=1}^m \left({-\tau_{zi} + 1}\right)}{\displaystyle \prod_{j=1}^n \left({\tau_{pj} + 1}\right)} e^{-\tau_0 s} \quad \textrm{ với } \tau_{p1} > \tau_{p2} > \tau_{p3} > \cdots
    \end{align*}

    \begin{itemize}
        \item Xấp xỉ về khâu quán tính bậc nhất có trễ với mô hình: $\tilde{G}(s) = \dfrac{k e^{-\theta s}}{\tau s + 1}$, trong đó:
            \begin{align*}
                \tau & = \tau_{p1} + \dfrac{\tau_{p2}}{2}; &
                \theta & = \tau_0 + \dfrac{\tau_{p2}}{2} + \sum_{j = 3}^n \tau_{pj} + \sum_{i = 1}^m \tau_{zi} \\
                & & & = \tau_0 + \dfrac{\tau_{p2}}{2} + \left({\tau_{p3} + \tau_{p4} + \cdots + \tau_{pn}}\right) + \left({\tau_{z1} + \tau_{z2} + \cdots + \tau_{zm}}\right)
            \end{align*}

        \item Xấp xỉ về khâu quán tính bậc hai có trễ với mô hình: $\tilde{G}(s) = \dfrac{k e^{-\theta s}}{\left({\tau_1 s + 1}\right) \left({\tau_2 s + 1}\right)}$, trong đó:
            \begin{align*}
                \tau_1 & = \tau_{p1}; & \tau_2 & = \tau_{p2} + \dfrac{\tau_{p3}}{2}; &
                \theta & = \tau_0 + \dfrac{\tau_{p3}}{2} + \sum_{j = 4}^n \tau_{pj} + \sum_{i = 1}^m \tau_{zi} \\
                & & & & & = \tau_0 + \dfrac{\tau_{p3}}{2} + \left({\tau_{p4} + \tau_{p5} + \cdots + \tau_{pn}}\right) + \left({\tau_{z1} + \tau_{z2} + \cdots + \tau_{zm}}\right)
                \end{align*}
    \end{itemize}

\section{Hàm truyền của bộ điều khiển PI và PID}
    \begin{itemize}
        \item Hàm truyền của bộ điều khiển PI cho khâu quán tính bậc nhất có dạng:
            \begin{align*}
                K(s) = K_P + \dfrac{K_I}{s} = K_P \left({1 + \dfrac{1}{T_I s}}\right)
            \end{align*}

        \item Hàm truyền của bộ điều khiển PID cho khâu quán tính bậc hai có dạng:
            \begin{align*}
                K(s) = K_P + \dfrac{K_I}{s} + K_D s = K_P \left({1 + \dfrac{1}{T_I s} + T_D s}\right)
            \end{align*}
    \end{itemize}

\section{Phương pháp xác định các thông số cuả bộ điều khiển PI và PID dựa trên mô hình mẫu}
\subsection{Phương pháp Haalman}
    \begin{itemize}
        \item Với khâu quán tính bậc nhất $\tilde{G}(s) = \dfrac{k e^{-\theta s}}{\tau s + 1}$ (mô hình FOPDT), sử dụng bộ điều khiển PI:
            \begin{align*}
                K(s) = K_P \left({1 + \dfrac{1}{T_I s}}\right) \quad \textrm{với} \quad K_P = \dfrac{2 \tau}{3 k \theta}; \quad T_I = \tau
            \end{align*}

        \item Với khâu quán tính bậc hai $\tilde{G}(s) = \dfrac{k e^{-\theta s}}{\left({\tau_1 s + 1}\right) \left({\tau_2 s + 1}\right)}$ (mô hình SOPDT), sử dụng bộ điều khiển PID:
            \begin{align*}
                K(s) = K_P \left({1 + \dfrac{1}{T_I s} + T_D s}\right) \quad \textrm{với} \quad K_P = \dfrac{2 \left({\tau_1 + \tau_2}\right)}{3 k \theta}; \quad T_I = \tau_1 + \tau_2; \quad T_D = \dfrac{\tau_1 \tau_2}{\tau_1 + \tau_2}
            \end{align*}
    \end{itemize}

\subsection{Phương pháp tổng hợp trực tiếp (Direct Synthesis -- DS, phương pháp chỉnh định Lamda)}
    \begin{itemize}
        \item Với khâu quán tính bậc nhất $\tilde{G}(s) = \dfrac{k e^{-\theta s}}{\tau s + 1}$ (mô hình FOPDT), sử dụng bộ điều khiển PI:
        \begin{align*}
            K(s) = K_P \left({1 + \dfrac{1}{T_I s}}\right) \quad \textrm{với} \quad K_P = \dfrac{\tau}{k\left({\tau_c + \theta}\right)}; \quad T_I = \tau
        \end{align*}

        với $\tau_c$ là hằng số thời gian quán tính.

        \item Với khâu quán tính bậc hai $\tilde{G}(s) = \dfrac{k e^{-\theta s}}{\left({\tau_1 s + 1}\right) \left({\tau_2 s + 1}\right)}$ (mô hình SOPDT), sử dụng bộ điều khiển PID:
            \begin{align*}
                K(s) = K_P \left({1 + \dfrac{1}{T_I s} + T_D s}\right) \quad \textrm{với} \quad K_P = \dfrac{\tau_1 + \tau_2}{k\left({\tau_c + \theta}\right)}; \quad T_I = \tau_1 + \tau_2; \quad T_D = \dfrac{\tau_1 \tau_2}{\tau_1 + \tau_2}
            \end{align*}

        với $\tau_c$ là hằng số thời gian quán tính.
    \end{itemize}

\subsection{Bảng tổng hợp xác định thông số của bộ điều khiển PI, PID theo phương pháp Haalman và phương pháp tổng hợp trực tiếp}

    \begin{table}[htp]
        \begin{center}
            \begin{tabular}{llll}
                \toprule
                \multicolumn{2}{c}{Khâu quán tính bậc nhất (FOPDT)} & \multicolumn{2}{c}{Khâu quán tính bậc hai (SOPDT)} \\
                \multicolumn{2}{c}{$\tilde{G}(s) = \dfrac{k e^{-\theta s}}{\tau s + 1}$} & \multicolumn{2}{c}{$\tilde{G}(s) = \dfrac{k e^{-\theta s}}{\left({\tau_1 s + 1}\right) \left({\tau_2 s + 1}\right)}$} \\
                \multicolumn{2}{c}{$K_{PI}(s) = K_P \left({1 + \dfrac{1}{T_I s}}\right)$} & \multicolumn{2}{c}{$K_{PID}(s) = K_P \left({1 + \dfrac{1}{T_I s} + T_D s}\right)$} \\
                \midrule
                Phương pháp Haalman & Phương pháp DS & Phương pháp Haalman & Phương pháp DS \\
                \midrule
                $K_P = \dfrac{2 \tau}{3 k \theta}$ & $K_P = \dfrac{\tau}{k\left({\tau_c + \theta}\right)}$  & $K_P = \dfrac{2 \left({\tau_1 + \tau_2}\right)}{3 k \theta}$ & $K_P = \dfrac{\tau_1 + \tau_2}{k\left({\tau_c + \theta}\right)}$ \\
                $T_I = \tau$ & $T_I = \tau$ & $T_I = \tau_1 + \tau_2$ & $T_I = \tau_1 + \tau_2$ \\
                & & $T_D = \dfrac{\tau_1 \tau_2}{\tau_1 + \tau_2}$ & $T_D = \dfrac{\tau_1 \tau_2}{\tau_1 + \tau_2}$ \\
                \bottomrule
            \end{tabular}
        \end{center}
    \end{table}
