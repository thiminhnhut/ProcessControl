\begin{exercise}
    Cho hệ thống như hình \ref{baitap1-1binhchua}: Biết lưu lượng ra $F_2$ tỉ lệ với chiều cao chất lỏng theo công thức $F_2 = R.h^{\sfrac{4}{3}}$ với $R$ là hằng số. Tiết diện của bồn chứa là $A$.
    \begin{figure}[htp]
        \begin{center}
            \subimport{../mohinhoalythuyet/images/}{baitap1-1binhchua.tex}
        \end{center}
        \caption{Hệ thống 1 bình chứa} \label{baitap1-1binhchua}
    \end{figure}

    \begin{enumerate}
        \item Xác định các biến vào, biến ra, biến điều khiển, biến cần điều khiển và biến nhiễu.
        \item Viết phương trình động học cho mức chất lỏng trong bồn chứa.
        \item Tuyến tính hóa phương trình xây dựng được xung quanh vị trí cân bằng dựa trên phương pháp khai triển Taylor.
        \item Xác định hàm truyền $G(s) = \dfrac{H(s)}{F_1(s)}$
        \item Thiết kế bộ điều khiển PI sử dụng phương pháp tổng hợp trực tiếp Direct Synthesis, biết hằng số thời gian của hệ kín là $\tau_c = 0.5$.
    \end{enumerate}
\end{exercise}
