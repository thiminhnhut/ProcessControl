\begin{exercise}
    Một quá trình bao gồm cả cảm biến và van điều khiển có thể được mô hình hoá bởi hàm truyền bậc 4 như sau:
        \begin{align*}
            G(s) = \dfrac{100(-0.1s + 1)}{(5s + 1)(3s + 1)(0.04s+1)(0.5s+1)}
        \end{align*}
    \begin{enumerate}
        \item Xấp xỉ hàm truyền theo quy tắc Skogestad:
            \begin{enumerate}
                \item Xấp xỉ hàm truyền $G(s)$ về dạng bậc nhất có trễ sử dụng quy tắc Skogestad.
                \item Xấp xỉ hàm truyền $G(s)$ về dạng bậc hai có trễ sử dụng quy tắc Skogestad.
            \end{enumerate}

        \item Thiết kế bộ điều khiển sử dụng phương pháp Haalman:
            \begin{enumerate}
                \item Thiết kế bộ điều khiển PI sử dụng phương pháp Haalman.
                \item Thiết kế bộ điều khiển PID sử dụng phương pháp Haalman.
            \end{enumerate}

        \item Thiết kế bộ điều khiển sử dụng phương pháp tổng hợp trực tiếp Direct Synthesis -- DS:
            \begin{enumerate}
                \item Thiết kế bộ điều khiển PI sử dụng phương pháp tổng hợp trực tiếp Direct Synthesis, biết hằng số thời gian của hệ kín là $\tau_c = 0.51$.

                \item Thiết kế bộ điều khiển PID sử dụng phương pháp tổng hợp trực tiếp Direct Synthesis, biết hằng số thời gian của hệ kín là $\tau_c = 0.51$.
            \end{enumerate}
    \end{enumerate}
\end{exercise}
