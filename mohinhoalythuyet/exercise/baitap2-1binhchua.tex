\paragraph{Giả thiết}
    Cho hệ thống như hình \ref{baitap2-1binhchua}: Trong đó $w_1$ là dòng lưu lượng vào $[m^3/s]$, $w_2$ là dòng lưu lượng ra $[m^3/s]$ và $h$ là chiều cao của mức chất lỏng $[m]$. Biết lưu lượng ra $w_2$ tỉ lệ với căn bậc hai của chiều cao mực chất lỏng bởi hằng số $C_v$. Diện tích mặt cắt ngang của bồn chứa là $A = 2[m^2]$. Khối lượng riêng của chất lỏng là $\rho = 500[kg/m^3]$.
    \begin{figure}[htp]
        \begin{center}
            \subimport{images/}{baitap2-1binhchua.tex}
        \end{center}
        \caption{Hệ thống 1 bình chứa} \label{baitap2-1binhchua}
    \end{figure}

\paragraph{Yêu cầu}
    \begin{enumerate}[a.]
        \item Xác định các biến vào, biến ra, biến điều khiển, biến cần điều khiển và biến nhiễu.
        \item Viết phương trình động học cho mức chất lỏng trong bồn chứa.
        \item Viết phương trình động học ở trạng thái ổn định mức. Biết trạng thái ổn định: $w_1 = 2,4~m^3/s$ và $h = 1,44~m$. Tìm $C_v$.
        \item Tuyến tính hóa phương trình xây dựng được xung quanh vị trí cân bằng dựa trên phương pháp khai triển Taylor.
        \item Xác định hàm truyền $G(s) = \dfrac{H(s)}{W_1(s)}$
    \end{enumerate}

\paragraph{Bài giải}
    \begin{enumerate}[\it a.]
        \item \textit{Xác định các biến vào, biến ra, biến điều khiển, biến cần điều khiển và biến nhiễu.}
            \begin{itemize}
                \item Biến vào: $w_1, w_2$.
                \item Biến ra: $h$.
                \item Biến điều khiển: $w_1$.
                \item Biến cần điều khiển: $h$.
                \item Biến nhiễu: $w_2$.
            \end{itemize}

        \item \textit{Viết phương trình động học cho mức chất lỏng trong bồn chứa.}
            \begin{itemize}
                \item Phương trình cân bằng vật chất:
                    \begin{align} \label{eq:baitap2-1binhchua}
                        \frac{dV}{dt} = w_1 - w_2 \Longleftrightarrow \dfrac{d\left({Ah}\right)}{dt} = w_1 - w_2 \Longleftrightarrow \dfrac{dh}{dt} = \dfrac{1}{A} \left({w_1 - w_2}\right)
                    \end{align}

                \item Thay $w_2 = C_v\sqrt{h}$ vào (\ref{eq:baitap2-1binhchua}), ta có:
                    \begin{align}
                        \dfrac{dh}{dt} = \dfrac{1}{A} \left({w_1 - w_2}\right) = \dfrac{1}{A} \left({w_1 - C_v\sqrt{h}}\right)
                    \end{align}

                \item Kết luận, phương trình mô tả quá trình:
                    \begin{align}
                        \dfrac{dh}{dt} = \dfrac{1}{A} \left({w_1 - C_v\sqrt{h}}\right)
                    \end{align}
            \end{itemize}

        \item \textit{Viết phương trình động học ở trạng thái ổn định mức. Biết trạng thái ổn định: $w_1 = 2,4~m^3/s$ và $h = 1,44~m$. Tìm $C_v$.}
            \begin{itemize}
                \item Gọi $\left({\overline{w_1}, \overline{h}}\right)$ là điểm làm việc cân bằng của hệ thống.

                \item Đặt $f\left({w_1, h}\right) = \dot{h} = \dfrac{1}{A} \left({w_1 - C_v\sqrt{h}}\right)$

                \item Tại điểm làm việc cân bằng $\left({\overline{w_1}, \overline{h}}\right)$ thì
                    \begin{align}
                        f\left({\overline{w_1}, \overline{h}}\right) = 0 \Longleftrightarrow \dfrac{1}{A} \left({\overline{w_1} - C_v\sqrt{\overline{h}}}\right) = 0
                    \end{align}

                \item Kết luận, phương trình động học ở trạng thái ổn định mức:
                    \begin{align} \label{eq:baitap2-1binhchua-2}
                        \dfrac{1}{A} \left({\overline{w_1} - C_v\sqrt{\overline{h}}}\right) = 0
                    \end{align}

                \item Thông số ở trạng thái ổn định: $\overline{w_1} = 2,4~m^3/s$ và $\overline{h} = 1,44~m$, nên thay vào phương trình (\ref{eq:baitap2-1binhchua-2}), ta có:
                    \begin{align}
                        \dfrac{1}{A} \left({\overline{w_1} - C_v\sqrt{\overline{h}}}\right) = 0 \Longleftrightarrow \dfrac{1}{2} \left({2,4 - C_v\sqrt{1,44}}\right) = 0 \Longleftrightarrow C_v = 2[m^2/s]
                    \end{align}
            \end{itemize}

        \item \textit{Tuyến tính hóa mô hình tại điểm làm việc cân bằng.}
            \begin{itemize}
                \item Gọi $w_1 = \overline{w_1} + \Delta w_1, h = \overline{h} + \Delta h$.

                \item Khai triển Taylor cho $f \left({w_1, h}\right) = \dot{h} = \dfrac{1}{A} \left({w_1 - C_v\sqrt{h}}\right)$, ta có:
                    \begin{align}
                        \dot{h} = \Delta \dot{h} & = f\left({\overline{w_1} + \Delta w_1, \overline{h} + \Delta h}\right) \\
                        & \approx \underbrace{f \left({\overline{w_1}, \overline{h}}\right)}_{0} + \left.\dfrac{\partial f}{\partial w_1}\right|_{\left({\overline{w_1}, \overline{h}}\right)} \Delta w_1 + \left.\dfrac{\partial f}{\partial h}\right|_{\left({\overline{w_1}, \overline{h}}\right)} \Delta h\\
                        & \approx \dfrac{1}{A} \left({\Delta w_1 - \dfrac{C_v}{2 \sqrt{\overline{h}}} \Delta h}\right)
                    \end{align}

                \item Kết luận, phương trình tuyến tính hóa của mô hình tại điểm làm việc cân bằng $\left({\overline{w_1}, \overline{h}}\right)$:
                    \begin{align}
                        \Delta \dot{h} = \dfrac{1}{A} \left({\Delta w_1 - \dfrac{C_v}{2 \sqrt{\overline{h}}} \Delta h}\right)
                    \end{align}
            \end{itemize}

        \item \textit{Xác định hàm truyền $G(s) = \dfrac{H(s)}{W_1(s)}$}
            \begin{itemize}
                \item Ta có: $\Delta \dot{h} = \dfrac{1}{A} \left({\Delta w_1 - \dfrac{C_v}{2 \sqrt{\overline{h}}} \Delta h}\right)$, thực hiện biến đổi Laplace 2 vế của phương trình ta có:
                    \begin{align}
                        & s H(s) = \dfrac{1}{A} \left[{W_1(s) - \dfrac{C_v}{2 \sqrt{\overline{h}}} H(s)}\right]\\
                        \Longleftrightarrow & s A H(s) + \dfrac{C_v}{2 \sqrt{\overline{h}}} H(s) = W_1(s)\\
                        \Longleftrightarrow & \left({s A + \dfrac{C_v}{2 \sqrt{\overline{h}}}}\right) H(s) = W_1(s) \\
                        \Longleftrightarrow & \dfrac{H(s)}{W_1(s)} = \frac{1}{s A + \dfrac{C_v}{2 \sqrt{\overline{h}}}}
                    \end{align}

                \item Kết luận:
                    \begin{align}
                        G(s) = \dfrac{H(s)}{W_1(s)} = \frac{1}{s A + \dfrac{C_v}{2 \sqrt{\overline{h}}}}
                    \end{align}
            \end{itemize}
    \end{enumerate}
