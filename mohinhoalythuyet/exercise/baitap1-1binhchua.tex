\paragraph{Giả thiết}
    Cho hệ thống như hình \ref{baitap1-1binhchua}: Biết lưu lượng ra $F_2$ tỉ lệ với chiều cao chất lỏng theo công thức $F_2 = R.h^{\sfrac{3}{2}}$ với $R$ là hằng số. Tiết diện của bồn chứa là $A$.
    \begin{figure}[htp]
        \begin{center}
            \subimport{images/}{baitap1-1binhchua.tex}
        \end{center}
        \caption{Hệ thống 1 bình chứa} \label{baitap1-1binhchua}
    \end{figure}

\paragraph{Yêu cầu}
    \begin{enumerate}[a.]
        \item Xác định các biến vào, biến ra, biến điều khiển, biến cần điều khiển và biến nhiễu.
        \item Viết phương trình động học cho mức chất lỏng trong bồn chứa.
        \item Tuyến tính hóa phương trình xây dựng được xung quanh vị trí cân bằng dựa trên phương pháp khai triển Taylor.
        \item Xác định hàm truyền $G(s) = \dfrac{H(s)}{F_1(s)}$
    \end{enumerate}

\paragraph{Bài giải}
    \begin{enumerate}[\it a.]
        \item \textit{Xác định các biến vào, biến ra, biến điều khiển, biến cần điều khiển và biến nhiễu.}
            \begin{itemize}
                \item Biến vào: $F_1, F_2$.
                \item Biến ra: $h$.
                \item Biến điều khiển: $F_1$ hoặc $F_2$.
                \item Biến cần điều khiển: $h$.
                \item Biến nhiễu: $F_2$ hoặc $F_1$.
            \end{itemize}

        \item \textit{Viết phương trình động học cho mức chất lỏng trong bồn chứa.}
            \begin{itemize}
                \item Phương trình cân bằng vật chất:
                    \begin{align} \label{eq:baitap1-1binhchua}
                        \frac{dV}{dt} = F_1 - F_2 \Longleftrightarrow \dfrac{d\left({Ah}\right)}{dt} = F_1 - F_2 \Longleftrightarrow \dfrac{dh}{dt} = \dfrac{1}{A} \left({F_1 - F_2}\right)
                    \end{align}

                \item Thay $F_2 = R.h^{\sfrac{3}{2}}$ vào (\ref{eq:baitap1-1binhchua}), ta có:
                    \begin{align}
                        \dfrac{dh}{dt} = \dfrac{1}{A} \left({F_1 - F_2}\right) = \dfrac{1}{A} \left({F_1 - R.h^{\sfrac{3}{2}}}\right)
                    \end{align}

                \item Kết luận, phương trình mô tả quá trình:
                    \begin{align}
                        \dfrac{dh}{dt} = \dfrac{1}{A} \left({F_1 - R.h^{\sfrac{3}{2}}}\right)
                    \end{align}
            \end{itemize}

        \item \textit{Tuyến tính hóa phương trình xây dựng được xung quanh vị trí cân bằng dựa trên phương pháp khai triển Taylor.}
            \begin{itemize}
                \item Gọi $\left({\overline{F_1}, \overline{h}}\right)$ là điểm làm việc cân bằng của hệ thống.

                \item Gọi $F_1 = \overline{F_1} + \Delta F_1, h = \overline{h} + \Delta h$.

                \item Đặt $f\left({F_1, h}\right) = \dot{h} = \dfrac{1}{A} \left({F_1 - R.h^{\sfrac{3}{2}}}\right)$
                    \begin{itemize}
                        \item Tại điểm làm việc cân bằng $\left({\overline{F_1}, \overline{h}}\right)$ thì
                            \begin{align}
                                f\left({\overline{F_1}, \overline{h}}\right) = 0 \Longleftrightarrow \dfrac{1}{A} \left({\overline{F_1} - R.\overline{h}^{\sfrac{3}{2}}}\right) = 0
                            \end{align}

                        \item Khai triển Taylor cho $f \left({F_1, h}\right) = \dot{h} = \dfrac{1}{A} \left({F_1 - R.h^{\sfrac{3}{2}}}\right)$, ta có:
                            \begin{align}
                                \dot{h} = \Delta \dot{h} & = f\left({\overline{F_1} + \Delta F_1, \overline{h} + \Delta h}\right) \\
                                & \approx \underbrace{f \left({\overline{F_1}, \overline{h}}\right)}_{0} + \left.\dfrac{\partial f}{\partial F_1}\right|_{\left({\overline{F_1}, \overline{h}}\right)} \Delta F_1 + \left.\dfrac{\partial f}{\partial h}\right|_{\left({\overline{F_1}, \overline{h}}\right)} \Delta h\\
                                & \approx \dfrac{1}{A} \left({\Delta F_1 - \dfrac{3}{2} R \overline{h}^{\sfrac{1}{2}} \Delta h}\right)
                            \end{align}

                        \item Thay $\Delta F_1 = F_1$ và $\Delta h = h$, ta có:
                            \begin{align}
                                \dfrac{dh}{dt} = \dfrac{1}{A} \left({F_1 - \dfrac{3}{2} R \overline{h}^{\sfrac{1}{2}} h}\right)
                            \end{align}
                    \end{itemize}

                \item Kết luận, phương trình tuyến tính hóa của mô hình tại điểm làm việc cân bằng $\left({\overline{F_1}, \overline{h}}\right)$:
                    \begin{align}
                        \dfrac{dh}{dt} = \dfrac{1}{A} \left({F_1 - \dfrac{3}{2} R \overline{h}^{\sfrac{1}{2}}h}\right)
                    \end{align}
            \end{itemize}

        \item \textit{Xác định hàm truyền $G(s) = \dfrac{H(s)}{F_1(s)}$}
            \begin{itemize}
                \item Ta có: $\dfrac{dh}{dt} = \dfrac{1}{A} \left({F_1 - \dfrac{3}{2} R \overline{h}^{\sfrac{1}{2}} h}\right)$, thực hiện biến đổi Laplace 2 vế của phương trình ta có:
                    \begin{align}
                        & s H(s) = \dfrac{1}{A} \left[{F_1(s) - \dfrac{3}{2} R \overline{h}^{\sfrac{1}{2}} H(s)}\right]\\
                        \Longleftrightarrow & s A H(s) + \dfrac{3}{2} R \overline{h}^{\sfrac{1}{2}} H(s) = F_1(s)\\
                        \Longleftrightarrow & \left({s A + \dfrac{3}{2} R \overline{h}^{\sfrac{1}{2}}}\right) H(s) = F_1(s) \\
                        \Longleftrightarrow & \dfrac{H(s)}{F_1(s)} = \frac{1}{s A + \dfrac{3}{2} R \overline{h}^{\sfrac{1}{2}}}
                    \end{align}

                \item Kết luận:
                    \begin{align}
                        G(s) = \dfrac{H(s)}{F_1(s)} = \frac{1}{s A + \dfrac{3}{2} R \overline{h}^{\sfrac{1}{2}}}
                    \end{align}
            \end{itemize}
    \end{enumerate}
